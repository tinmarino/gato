%%% ------------------------------------------------------------
%%% BEGIN PREAMBLE
%%% ------------------------------------------------------------
\documentclass[paper=a4,fontsize=12pt]{scrartcl}	 			% KOMA-article class
\usepackage[english]{babel}
%%\usepackage[francais]{babel}
\usepackage[T1]{fontenc}							% For true french font 
\usepackage{lmodern}								% For scalable font in T1 (instead of OT1).
\usepackage[utf8]{inputenc}
\DeclareUnicodeCharacter{00A0}{ } % no utf8 pb 

\linespread{1.2}  								%% interligne
\textheight=700px								% Saving trees ;-) 

\usepackage[protrusion=true,expansion=true]{microtype}				% Better typography
\usepackage[pdftex]{graphicx}							% Enable pdflatex
\usepackage[svgnames]{xcolor}							% Colors by their 'svgnames'

\usepackage[bottom=0in,top=1.5in]{geometry}					% Geometry
	\addtolength{\oddsidemargin}{-0.25in}
	\addtolength{\evensidemargin}{-0.25in}
        \addtolength{\textwidth}{0.5in}
        \addtolength{\textheight}{3in}
        	
\usepackage{url}								% Clickable URL's
\usepackage{wrapfig}								% Wrap text along figures
\frenchspacing									% Better looking spacings after periods
\pagestyle{empty}								% No pagenumbers/headers/footers
%\usepackage{bbding}								% Symbols


% Martin packages
\usepackage[absolute]{textpos}
    \setlength{\TPHorizModule}{1mm} % sets our horizontal unit of measuring
    \setlength{\TPVertModule}{1mm} % sets our vertical unit of measuring
    \textblockorigin{0mm}{0mm} % and we start measuring in the top left corner



%%% Custom sectioning (sectsty package)
%%% ------------------------------------------------------------
\usepackage{sectsty}							% Custom sectioning (see below)

\sectionfont{%								% Change font of \section command
	\usefont{OT1}{phv}{b}{n}%					% bch-b-n: CharterBT-Bold font
	\sectionrule{0pt}{0pt}{-5pt}{3pt}
	}
	

	
	
	
%%% Macros
%%% ------------------------------------------------------------
\newlength{\spacebox}
\settowidth{\spacebox}{8888888888}				% Box to align text
\newcommand{\sepspace}{\vspace*{1em}}			% Vertical space macro


% \newcommand{\MyName}[1]{
% 		\Huge \usefont{OT1}{phv}{b}{n} \hfill #1 		% Name
% 		\par \normalsize \normalfont}

\newcommand{\MyName}[1]{ %Name
%       \begin{center}
   \begin{textblock}{100}(70,20)\centering
     % (x,y) les coord du cntre 
     % 8.5 = width of the block
     % 0 = zero horizontal units away from our origin
     % 4 = four vertical units away from our origin
          \LARGE\bfseries {#1}
   \end{textblock}  
%       \end{center}
}


\newcommand{\MyFoto}[1]{ % path of the foto  
  \begin{textblock}{210}(170,2)
		\includegraphics[width=3cm]{#1}
  \end{textblock}  
   }
   
\newcommand{\MyData}[1]{ % string of the personal data 
    \begin{textblock}{210}(5,2)
      { \vbox to 2cm { \vfill  { \flushleft  #1 }  }  }
    \end{textblock}  
    }
      
      
\newcommand{\NewPart}[1]{\section*{\uppercase{#1}}}

\newcommand{\PersonalEntry}[2]{
		\noindent\hangindent=1em\hangafter=0 		% Indentation
		\parbox{\spacebox}{						% Box to align text
		   \textit{#1}}								% Entry name (birth, address, etc.)
		 #2 \par}					% Entry value

\newcommand{\SkillsEntry}[2]{						% Same as \PersonalEntry
		\noindent\hangindent=2em\hangafter=0 		% Indentation
		\parbox{\spacebox}{						% Box to align text
		\textit{#1}}								% Entry name (birth, address, etc.)
		\hspace{1.5em} #2 \par}					% Entry value	
		
\newcommand{\EducationEntry}[4]{
		\noindent \textbf{#1} \hfill 					% Study
		\colorbox{Black}{%
			\parbox{4.5em}{%
			\hfill\color{White}#2}} \par				% Duration
		\noindent \textit{#3} \par					% School
		\noindent\hangindent=2em\hangafter=0 \small #4 	% Description
		\normalsize \par}

\newcommand{\WorkEntry}[4]{						% Same as \EducationEntry
		\noindent \textbf{#1} \hfill 					% Jobname
		\colorbox{Black}{\color{White}#2} \par		% Duration
		\noindent \textit{#3} \par					% Company
		\noindent\hangindent=2em\hangafter=0 \small #4 	% Description
		\normalsize \par}



		
		
		
	

%%% ------------------------------------------------------------
%%% BEGIN DOCUMENT
%%% ------------------------------------------------------------
\begin{document}


\MyName{Roc\'io Sanhueza Repetto}
\MyFoto{Figure/Rocio1.png}
\MyData{
  \PersonalEntry{Née le :}   {4 Août 1989} 
  \PersonalEntry{Email :}{\url{r.sanhuezarepetto@gmail.com}}
  \PersonalEntry{Nationalité :}  {Chileno-Italienne} 
  \PersonalEntry{Tel :}  {07 83 88 33 32} 
}




%%% 1/  Education
%%% ------------------------------------------------------------
\NewPart{Formation}{} 


\EducationEntry{Master 1 - Science Politique }{2015-2016}
    {Université de Rennes 1, Rennes, France}
    {}


\EducationEntry{Maîtrise en communication sociale - titre de journaliste}{2008-2013}
    {Universidad de Santiago de Chile, Santiago, Chili}
    {}

\EducationEntry{\'Echange universitaire - journalisme}{2011}
    {Universidade Estadual Paulista “Julio de Mesquita Filho”(UNESP), Sao Paulo, Brésil}
    {}



\EducationEntry{Lycée}{2004-2007}
  {Liceo Augusto D'Halmar, Santiago, Chili}

  
  
%%% 2/ WORK EXPERIENCE
%%% ------------------------------------------------------------
\NewPart{Expérience Professionnelle}{}


\EducationEntry{Conseillère communicationnelle du sénateur Felipe Harboe}{2014}{Sénat du Chili}
{Gestion communicationelle au sénat.} 

\EducationEntry{Productrice générale}{2013-2014}{Aluro procucciones, Santiago, Chili}
{ Production générale de la première et deuxième saison du programme culturel 
   “La Bicicleta, cultura entre ruedas” transmit par le canal 13C. 
}

\EducationEntry{Journaliste}{2013}{M\'as comunicaciones, Santiago, Chili}
{Gestion de presse pour des marques comme Dole, Artilec, MOR et Flores.}

\EducationEntry{Chargée de projet}{2012}{Killalue Corporaci\'on, Santiago, Chili}
{Recherche en affaires légales et environnementales :
  rédaction d'actes, maintenance de site web.
}





%%% 3/ Skills
%%% ------------------------------------------------------------
\NewPart{Compétences}{}

\SkillsEntry{Langues}{Espagnol (langue maternelle)}
\SkillsEntry{}{Français (courant)}
\SkillsEntry{}{Anglais (courant)}
\SkillsEntry{}{Portugais (courant)} 

\SkillsEntry{Softwares}{Microsoft Word, Excel, Power Point, Adobe Photoshop, Premiere.}



\end{document}