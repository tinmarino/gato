%%% LaTeX Template: Curriculum Vitae
%%%
%%% Source: http://www.howtotex.com/
%%% Feel free to distribute this template, but please keep the referal to HowToTeX.com.
%%% Date: July 2011

%%% ------------------------------------------------------------
%%% BEGIN PREAMBLE 
%%% ------------------------------------------------------------
\documentclass[paper=a4,fontsize=11pt]{scrartcl}	 			% KOMA-article class
\usepackage[utf8]{inputenc}
\usepackage[francais]{babel}

\usepackage[protrusion=true,expansion=true]{microtype}		% Better typography
\usepackage{amsmath,amsfonts,amsthm}					% Math packages
% \usepackage[pdftex]{graphicx}								% Enable pdflatex
\usepackage[svgnames]{xcolor}							% Colors by their 'svgnames'

\usepackage[bottom=0in,top=1.5in]{geometry}
	\addtolength{\oddsidemargin}{-0.25in}
	\addtolength{\evensidemargin}{-0.25in}
        	\addtolength{\textwidth}{0.5in}
        	
        		
        	\addtolength{\textheight}{3in}
        	
\textheight=700px									% Saving trees ;-) 
\usepackage{url}										% Clickable URL's
\usepackage{wrapfig}									% Wrap text along figures

\frenchspacing									% Better looking spacings after periods
\pagestyle{empty}								% No pagenumbers/headers/footers
%\usepackage{bbding}									% Symbols

% Martin packages

\usepackage[absolute]{textpos}
    \setlength{\TPHorizModule}{1mm} % sets our horizontal unit of measuring
    \setlength{\TPVertModule}{1mm} % sets our vertical unit of measuring
    \textblockorigin{0mm}{0mm} % and we start measuring in the top left corner



%%% Custom sectioning (sectsty package)
%%% ------------------------------------------------------------
\usepackage{sectsty}							% Custom sectioning (see below)

\sectionfont{%									% Change font of \section command
	\usefont{OT1}{phv}{b}{n}%					% bch-b-n: CharterBT-Bold font
	\sectionrule{0pt}{0pt}{-5pt}{3pt}
	}
	

	
	
	
%%% Macros
%%% ------------------------------------------------------------
\newlength{\spacebox}
\settowidth{\spacebox}{8888888888}				% Box to align text
\newcommand{\sepspace}{\vspace*{1em}}			% Vertical space macro


% \newcommand{\MyName}[1]{
% 		\Huge \usefont{OT1}{phv}{b}{n} \hfill #1 		% Name
% 		\par \normalsize \normalfont}

\newcommand{\MyName}[1]{ %Name
%       \begin{center}
   \begin{textblock}{100}(70,20)\centering
     % (x,y) les coord du cntre 
     % 8.5 = width of the block
     % 0 = zero horizontal units away from our origin
     % 4 = four vertical units away from our origin
          \huge\bfseries {#1}
   \end{textblock}  
%       \end{center}
      }


\newcommand{\MyFoto}[1]{ % path of the foto  
	\begin{textblock}{210}(170,2)
		\includegraphics[width=3cm]{#1}
	\end{textblock}  
	}
   
\newcommand{\MyData}[1]{ % string of the personal data 
    \begin{textblock}{210}(5,2)
      { \vbox to 2cm { \vfill  { \flushleft  #1 }  }  }
    \end{textblock}  
    }
      
      
      
      
      
\newcommand{\NewPart}[1]{\section*{\uppercase{#1}}}



\newcommand{\PersonalEntry}[2]{
		\noindent\hangindent=1em\hangafter=0 		% Indentation
		\parbox{\spacebox}{						% Box to align text
		   \textit{#1}}								% Entry name (birth, address, etc.)
		 #2 \par}					% Entry value

\newcommand{\SkillsEntry}[2]{						% Same as \PersonalEntry
		\noindent\hangindent=2em\hangafter=0 		% Indentation
		\parbox{\spacebox}{						% Box to align text
		\textit{#1}}								% Entry name (birth, address, etc.)
		\hspace{1.5em} #2 \par}					% Entry value	
		
\newcommand{\EducationEntry}[4]{
		\noindent \textbf{#1} \hfill 					% Study
		\colorbox{Black}{%
			\parbox{4.5em}{%
			\hfill\color{White}#2}} \par				% Duration
		\noindent \textit{#3} \par					% School
		\noindent\hangindent=2em\hangafter=0 \small #4 	% Description
		\normalsize \par}

\newcommand{\WorkEntry}[4]{						% Same as \EducationEntry
		\noindent \textbf{#1} \hfill 					% Jobname
		\colorbox{Black}{\color{White}#2} \par		% Duration
		\noindent \textit{#3} \par					% Company
		\noindent\hangindent=2em\hangafter=0 \small #4 	% Description
		\normalsize \par}



		
		
		
	
%\renewcommand{\familydefault}{\sfdefault}
%\renewcommand{\familydefault}{}
%     \renewcommand{\sfdefault}{ptm}
%     \renewcommand{\familydefault}{\sfdefault}
    
% \usepackage[active,tightpage]{preview}

% \renewcommand{\PreviewBorder}{1in}


%%% ------------------------------------------------------------
%%% BEGIN DOCUMENT
%%% ------------------------------------------------------------
\begin{document}
% \begin{preview}



% \begin{wrapfigure}{l}{0.5\textwidth}
% 	\vspace*{-2em}
% 		\includegraphics[width=0.15\textwidth]{photo}
% \end{wrapfigure}

\MyName{Martin Tourneboeuf}
% \sepspace

\MyFoto{../Figure/martin2015.jpg}


\MyData
{
\PersonalEntry{Né le :}   {25 Décembre 1988} 
\PersonalEntry{Email :}{\url{martin.tourneboeuf@gmail.com}}
\PersonalEntry{Tel :}  {02 33 54 69 46} 
}





% \begin{tabular}{p{7cm}p{6cm}}
% 
%      & December 25, 1988 \\
%    
%      & \url{tinmarino@gmail.com}%
%     \end{tabular}
    

  
  
  
% \MyBirth}{December 25, 1988} 
% % \PersonalEntry{Address}{}
% % \PersonalEntry{Phone}{}
% \MyMail{Mail}{\url{tinmarino@gmail.com}}


%%% Education
%%% ------------------------------------------------------------
\NewPart{Formation}{} 

% \EducationEntry{MSc. Name of Education}{2010-2012}{Name of University}{Descriptive text goes here


\EducationEntry{Master d'Astrophysique}{2011-2013}{Pontificia Universidad Cat\'olica (PUC), Santiago, Chili}
    { L'université Chilienne d'astronomie. 
    Le Chili est un leader mondial dans l'observation astronomique. }

\EducationEntry{Diplôme d'ingénieur}{2008-2013}{\'Ecole Polytechnique (X), Paris, France}
    {Enseignement d'ingénieur pluridisciplinaire puis spécialisation en Physique 
    ``des particules aux étoiles''.}



\EducationEntry{Math Sup - Math Spé}{2006-2008}{Lyc\'ee Victor Grignard, Cherbourg, France}
{Math, Physique, Sciences de l'Ingénieur (MPSI);
 deux ans de classes préparatoires aux concours des grandes écoles (CPGE).
}


\EducationEntry{Baccalaur\'eat}{2003-2005}{Lyc\'ee Victor Grignard, Cherbourg, France}
{Baccalauréat série S mention très bien. 
}



%%% Work experience
%%% ------------------------------------------------------------
\NewPart{Expérience Professionnelle}{}


\EducationEntry{Programmeur Java}{2014}{Santiago, Chili}
{ Développement de jeux vidéos, applications android (Java SDK, NDK). 
}

\EducationEntry{Serveur-Barman}{2014}{Insert-Coin, Santiago, Chili}
{ Serveur dans un restaurant avec consoles de jeu vidéo. 
}

\EducationEntry{Astronome}{2013}{European Southern Observatories (ESO), Santiago, Chili}
{Développement de ``ABISM'', un GUI en Python pour mesurer la qualité d'une image.
 Le but étant d'aider l'observateur à changer de routine d'optique adaptative sur NaCo, VLT.
}

\EducationEntry{Astronome}{2012}{Canada France Hawaii Telescope (CFHT, Waimea, USA}
{ 
    Etude statistique de l'histoire de formation stéllaire au sein des galaxies. 
}


\EducationEntry{Commandos Marine}{2008-2009}{Base des Fusiliers marins, Lorient, France}
{Huit mois d'entraînement au commandement inclus dans le cursus scolaire de Polytechnique.}

\EducationEntry{Ostreiculture}{2006-2008}{St-Vaast-la-Houge, France}
{Travail d'été consistant à retourner des poches à huître pendant la marée basse. 
}


\EducationEntry{Animateur}{2006-2008}{Camp de vacances, France}
{Cinq camps de vacances en tant qu'animateur (BAFA).}





%%% Skills
%%% ------------------------------------------------------------
\NewPart{Compétences}{}

\SkillsEntry{Langues}{Français (langue maternelle)}
\SkillsEntry{}{Espagnol (courant)}
\SkillsEntry{}{Anglais (courant)} 

\SkillsEntry{Softwares}{Python, 
         BaSh, 
         C, 
         Java,
         IRAF, 
         Assembly (NASM, GAS), 
         Vim, 
         \LaTeX. 
         }

\SkillsEntry{Activités}{Sport, voyage, programmation, bricolage}

%%% References
%%% ------------------------------------------------------------
% \NewPart{References}{}
% Available upon request


% \end{preview}
\end{document}
