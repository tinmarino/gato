\documentclass{tccv}
\usepackage[english]{babel}
\usepackage[utf8]{inputenc}							% For easy input with é 
\usepackage[T1]{fontenc}							% For true french font 
\usepackage{lmodern}								% For scalable font in T1 (instead of OT1).
\usepackage{amsfonts} 								% for the \checkmark command 



\begin{document}

\part{Roc\'io Sanhueza}


\personal
    []
    {16 Rue Chicogné\newline 35000 -- Rennes}
    {+33 07 83 88 33 32}
    {r.sanhuezarepetto@gmail.com}



\section{Experience Professionnelle}


     

\begin{eventlist}

\item{Avril 2014 -- Mars 2015}
     {Sénat du Chili, Santiago, Chili}
     {Conseillère en communication et attachée de presse}

     % Missions
    \begin{itemize}
      \cvitem[\checkmark] Chargée de la mise en œuvre des actions de communication et des relations avec la presse nationale et régionale
      \cvitem[\checkmark] Rédaction des articles, communiqués et dossiers de presse. Gestion et supervision des interviews avec des chaînes de télévision
      \cvitem[\checkmark] Organisation des conférences de presse
      \cvitem[\checkmark] Administration de réseaux sociaux conjointement avec l’administrateur du site web
    \end{itemize}     
     


\item{Mai 2013 -- Déc. 2014}
     {Aluro producciones, Santiago, Chili}
     {Productrice générale}
    
    \begin{itemize}
      \cvitem[\checkmark] Gérer les rapports avec la chaîne de télévision et négociation avec eux les budgets. Recherche des marques sponsors
      \cvitem[\checkmark] Préparation du lieu de tournage, mobilisation, catering et interviewés de chaque chapitre. Aide à la préparation du contenu des entretiens
      \cvitem[\checkmark] Administration de réseaux sociaux et animation de communautés (Facebook, Twitter, Instagram)
      \cvitem[\checkmark] Création du dossier de presse pour les lancements des deux premières sessions du programme

    \end{itemize}     

\item{Oct. 2013 -- Fév. 2014 }     
  {Agence de communication Más comunicaciones}     
  {Journaliste}

\begin{itemize}
  \cvitem[\checkmark] Préparation des articles pour les médias
  \cvitem[\checkmark] Création des dossiers de presse 
  \cvitem[\checkmark] Cibler les journalistes spécialisés et améliorer régulièrement la basse de donnés. 
  \cvitem[\checkmark] Relances téléphoniques
\end{itemize}       



\item{Avril 2012 -- Avril. 2013 }     
  {Killahue corporation}     
  {Chargée de projet}

\begin{itemize}
  \cvitem[\checkmark] Recherche des informations juridiques et environnementales des projets similaires dans le cadre national ou internationale
  \cvitem[\checkmark] Gestion et coordination des réunions chaque semaine avec les membres de la corporation, plus la préparation des matériaux visuel et écrit
  \cvitem[\checkmark] Administration de réseaux sociaux 
  \cvitem[\checkmark] Formalisation et rédaction du travail réalisé et des décisions pris dans l’ensemble des réunions

\end{itemize}      
    

\item{Janv. 2012 -- Mars 2012 }     
  {Terra Networks Chili}     
  {Journaliste stagiaire – section économie}

\begin{itemize}
  \cvitem[\checkmark] Rédaction des articles et notes d’économie national et internationale
  \cvitem[\checkmark] Traduction des nouvelles du portugais ou anglaise au espagnol
  \cvitem[\checkmark] Veille et mise à jour du site économique
  \cvitem[\checkmark] Réalisation des interviews et couverture médiatique des thèmes économiques nationaux

\end{itemize}        
   
   


\end{eventlist}



\section{Education}

\begin{yearlist}

\item[Master 1 Science politique]{2015 -- 2016}
     {Université de Rennes 1}
     {Enseignements suivis: pensée politique contemporaine, régimes contemporaines, sociologie de la communication, pensée sociologique, Approches de lUnion Européen, Grand dossiers de l\' administration.}


  

\item[Diplôme en Communication sociale et journalisme]{2008 -- 2013}
     {Universidad de Santiago de Chile}
     {Mention très bien
      Spécialité politique
      (Bac+5)}
   
     
\item[Échange universitaire -- journalisme]{2011}
     {Universidade Estadual Paulista (UNESP)}
     {Enseignements suivis: réalité socio-économique et politique brésilienne, langue portugaise: littérature, sémiotique, stratégies de communication publique}


\end{yearlist}



\section{Compétences linguistiques}

\begin{factlist}
\item{Espagnol}{langue maternelle}	
\item{Français}{courant}	
\item{Anglais}{niveau B2}	
\item{Portugais}{niveau B2}
\end{factlist}

\section{Logiciels}

\begin{factlist}

\item{}{Microsoft Word, Excel, Power Point, Adobe Photoshop, Premiere, \\
Environnement PC et Linux,
Pack office et libre office, \\
%Python, 
Gimp,
HTML5,
\LaTeX.
}


\end{factlist}

\end{document}
