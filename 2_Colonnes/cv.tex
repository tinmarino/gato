\documentclass[30pt, french]{tccv}
\usepackage[hmargin=0.2cm,vmargin=0cm]{geometry}
\usepackage{amsfonts} 								% for the \checkmark command 
\usepackage[framemethod=TikZ]{mdframed}						% For box around text

\usepackage[overlay,absolute]{textpos}
\setlength{\TPHorizModule}{1cm}
\setlength{\TPVertModule}{1cm}

\usepackage{fontspec}
\usepackage{enumitem}




\mdfsetup{
   middlelinecolor=mybluegrey,
   middlelinewidth=1pt,
   backgroundcolor=white!0,
   roundcorner=10pt,
   skipabove=30,
   fontcolor=black,
   innertopmargin=-10pt
}

\fontdir[fonts/]
\colorlet{awesome}{awesome-skyblue}
%\defaultfontfeatures{Ligatures=TeX,Color=text}

\begin{document}
\begin{upshape}
\fontsize{9pt}{1em}\color{text}\selectfont

%%%%%%%%%%%%%%%%%%%%%%%%%%%%%%%%%%%%%%%%%%%%%%%%%%%%%%%%%%%%%%%%%%%%%%%%%%%%%%%%%%%%%%%%%%%%%%%%%%%%%%%%%%%%%%%%%%%%




\begin{textblock}{6.5}(0.5,0.5)
\personal
    []
    {16 Rue Chicogné, 35000 Rennes}
    {+33 07 83 88 33 32}
    {r.sanhuezarepetto@gmail.com}
\end{textblock}

\begin{textblock}{21}(0,1)
     \centering{\headerfirstnamestyle{Roc\'io}   \headerlastnamestyle{Sanhueza}}
\end{textblock}

\begin{textblock}{21}(17.5,0.5)
		\includegraphics[width=3cm]{../Figure/Rocio3.png}
\end{textblock}  



\begin{textblock}{21}(0,2.3)

\begin{center}
\fontsize{10pt}{1.5em}\color{text}\bodyfontlight\upshape\selectfont

	{\fontsize{14pt}{5em}\scshape\bfseries Journaliste attachée de presse avec double formation \\} 

	\vspace{5pt}
Connaissances développées en communication sociale et science politique 			 \\
Expériences dans les/le milieu culturel télévisuelle et politique parlementaire 		 \\
Mise en place des stratégies de communication et administration de réseaux sociaux 		\\

\end{center}
\end{textblock}  






\begin{textblock}{7}(0.5,4)
\begin{mdframed}

\section{Éducation}
\begin{yearlist}
\vspace{0.5cm}
\item[Master 1 Science politique]{2015 -- 2016}
     {Université de Rennes 1}
     {Enseignements suivis: pensée politique contemporaine, 
     régimes contemporaines, sociologie de la communication, pensée sociologique, 
     Appro\-ches de l'Union Européen, Grand dossiers de\- l'ad\-mi\-ni\-stra\-tion.}



\vspace{0.5cm}
\item[Diplôme en Communication sociale et journalisme (Bac+5)]{2008 -- 2013}
     {Universidad de Santiago de Chile}
     {Spécialité politique
     Mention très bien
     }

 \vspace{0.5cm}    
\item[Échange universitaire -- journalisme]{2011 -- 2011}
     {Universidade Estadual Pau\-li\-sta}
     {Enseignements suivis: réalité socio - économique et politique brésilienne. \\
     Langue portugaise: littérature, sémiotique, stratégies de communication publique}


\end{yearlist}
\end{mdframed}

\vspace{0.5cm}

\begin{mdframed}
\section{Compétences linguistiques}

\begin{factlist}
\item{Espagnol} {Langue maternelle}	
\item{Français} {Courant}	
\item{Anglais}  {Niveau B2}	
\item{Portugais}{Niveau B2}
\end{factlist}

\vspace{0.5cm}
\section{Logiciels}
Environnement PC et Linux,
Pack office et Libre office,
Adobe Photoshop, Premiere, \\
%Python, 
Gimp,
HTML5,
\LaTeX.

\vspace{0.5cm}
\section{Aptitudes}
\begin{itemize}[leftmargin=13pt]
  \setlength\itemsep{-3pt} 
  \cvitem[\checkmark]  Aisance communicationnelle et relationnelle 
  \cvitem[\checkmark]  Disciplinée et organisée 
  \cvitem[\checkmark]  Esprit d'équipe 
\end{itemize}



%%%%%%%%%%%%%%%%%%%%%%%%%%%%%%%%%%%%%%%%%%%%%%%%%%%%%%%%%%%%%%%%%%%%%%%%%%%%%%%%%%%%%%%%%%%%%%%%%%%%%%%%%%%%%%%%%%%%%%%%%%%%%%%%%%%%%%%%%%
%&
\end{mdframed}
\end{textblock}


\mdfsetup{middlelinecolor=white!0}

\begin{textblock}{13}(7.7,4)
\begin{mdframed}
\section{Expérience Professionnelle}


\begin{eventlist}

\setlength{\parskip}{0pt}
\item{Avril 2014 -- Mars 2015}
     {Sénat du Chili}
     {Conseillère en communication et attachée de presse}
     \fontsize{9pt}{1em}\color{text}\bodyfontlight\upshape\selectfont
     \makebox[1.4cm][l]{Contexte :} Le cabinet politique du sénateur Felipe Harboe cherche continuement améliorer sa visibilité et présence dans les médias \\ 
     \makebox[1.4cm][l]{Missions :} 

    \setlength{\parskip}{-10pt}
    \begin{itemize}
      \setlength\itemsep{-3pt} 
      \cvitem[\checkmark] Mise en œuvre des actions de communication et des relations avec la presse
      \cvitem[\checkmark] Rédaction des articles, communiqués et dossiers de presse. Supervision des interviews avec les médias
      \cvitem[\checkmark] Organisation de conférences de presse
      \cvitem[\checkmark] Administration de réseaux sociaux
    \end{itemize}     
    \makebox[1.4cm][l]{Résultats :} Présence constante dans la presse nationale et régionale  \\
    \makebox[1.4cm][l]{}          Entre 600 et 1100 visites quotidiennes sur le site web 
               
\setlength{\parskip}{0pt}        
\item{Mai 2013 -- Déc. 2014}
     {Aluro 35, Santiago, Chili}
     {Productrice générale}
     \fontsize{9pt}{1em}\color{text}\bodyfontlight\upshape\selectfont
    \makebox[1.4cm][l]{Contexte :} La maison de production indépendante Aluro 35 doit produire le programme culturel de télévision La Bicicleta transmit par la chaîne 13C \\
    \makebox[1.4cm][l]{Missions :}
    
    \setlength{\parskip}{-10pt}
    \begin{itemize}
      \setlength\itemsep{-3pt} 
      \cvitem[\checkmark] Gestion des rapports avec la chaîne et négociation de budgets                       
      \cvitem[\checkmark] Préparation du lieu de tournage, mobilisation et interviews. Aide à la préparation du contenu des entretiens 
      \cvitem[\checkmark] Administration de réseaux sociaux et animation de communautés (Facebook, Twitter, Instagram)                 
      \cvitem[\checkmark] Création du dossier de presse pour les lancements du programme                                               
    \end{itemize}     
\makebox[1.4cm][l]{Résultats:} Renouvellement d'une deuxième saison avec plus de sponsors \\
\makebox[1.4cm][l]{}	       Montée de la visibilité des artistes et associations culturels interviewés \\
\makebox[1.4cm][l]{}           Obtention d’un fond public du Ministère de Culture pour une troisième saison  \\


\setlength{\parskip}{0pt}    
\item{Oct. 2013 -- Fév. 2014 }     
  {Más comunicaciones, Santiago, Chili}     
  {Journaliste}
     \fontsize{9pt}{1em}\color{text}\bodyfontlight\upshape\selectfont
\makebox[1.4cm][l]{Contexte :} L’agence Mas comunicaciones doit en permanence améliorer la communication globale et la présence dans les médias
pour des marques comme Dole, Artilec, MOR et Flores\\
\makebox[1.4cm][l]{Missions :}

\setlength{\parskip}{-10pt}
\begin{itemize}
      \setlength\itemsep{-3pt} 
      \cvitem[\checkmark]  Préparation d'articles pour les médias                                            
      \cvitem[\checkmark]  Création de dossiers de presse                                                     
      \cvitem[\checkmark]  Ciblage des journalistes spécialisés et amélioration de la basse de donnés  
\end{itemize}       
\makebox[1.4cm][l]{Résultats:} Présence constante dans la presse nationale (El Mostrador; revues Paula, Mujer, Vanidades, Ya) et regionale (El Rancaguino) 

\setlength{\parskip}{0pt}
\item{Avril 2012 -- Avril 2013 }     
  {Killahue corporation, Santiago, Chili}     
  {Chargée de projet}
\fontsize{9pt}{1em}\color{text}\bodyfontlight\upshape\selectfont
\makebox[1.4cm][l]{Contexte :} Projet innovateur qui cherche à urbaniser un secteur rural du Chili, tout en préservant son milieu naturel et le développement de leurs habitants \\
\makebox[1.4cm][l]{Missions :}
     
\setlength{\parskip}{-10pt}
\begin{itemize}
      \setlength\itemsep{-3pt} 
      \cvitem[\checkmark] Recherche des informations juridiques et environnementales de projets similaires          
      \cvitem[\checkmark] Gestion et coordination de réunions, plus la préparation de matériaux visuels et écrits   
      \cvitem[\checkmark] Administration de réseaux sociaux                                                                    
      \cvitem[\checkmark] Formalisation et rédaction du travail réalisé et des décisions pris dans l’ensemble des réunions     
\end{itemize}      

\makebox[1.4cm][l]{Résultats:}  Définition de l’image corporatif \\
\makebox[1.4cm][l]{}            Identification et résolution des problématiques internes de la corporation \\
\makebox[1.4cm][l]{} 		Amélioration du niveau de présence dans les médias spécialisés \\


  
\setlength{\parskip}{0pt}
\item{Janv. 2012 -- Mars 2012 }     
  {Terra Networks, Santiago, Chili}     
  {Journaliste stagiaire – section économie}
  \fontsize{9pt}{1em}\color{text}\bodyfontlight\upshape\selectfont

\makebox[1.4cm][l]{Contexte :} Filiale Chilienne des contenus d’internet Terra  \\
\makebox[1.4cm][l]{Missions :}
  
\setlength{\parskip}{-10pt}
\begin{itemize}
      \setlength\itemsep{-3pt} 
      \cvitem[\checkmark] Rédaction d'articles et notes d’économie national et internationale
      \cvitem[\checkmark] Traduction de nouvelles du portugais ou de l'anglais à l'espagnol
      \cvitem[\checkmark] Veille et mise à jour du site économique
      \cvitem[\checkmark] Réalisation d'interviews et couverture médiatique des thèmes économiques nationaux
\end{itemize}        

\makebox[1.4cm][l]{Résultats:} Présence assurée du portail économique dans l’ensemble des nouvelles
\makebox[1.4cm][l]{}           Amélioration de l’interactivité des articles avec des infographies

   


\end{eventlist}


\end{mdframed}
\end{textblock}

\end{upshape}
\end{document}
