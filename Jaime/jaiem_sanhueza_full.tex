% -2/ Language, toremove
\def\tbflanguage{spanish}
\def\tbfbordertop{1}
\def\tbfborderleft{0.7}

% -1/ Type of CV
%\documentclass[paper=a4, fontsize=11pt]{tccv}          % Unfortynately must keep some stuff in class for now

% 0/ Default settings
\providecommand{\tbfbordertop}{0.6}
\providecommand{\tbfborderleft}{0.6}
\providecommand{\tbflanguage}{english}
\providecommand{\tbfcmd}{two_column}

% API
\usepackage{xargs}                          % Key - value parameters
\usepackage{xcolor}
\usepackage{pgf}                            % Math macro
\usepackage{xifthen}                        % If, else block

% Switch Html / Pdf for mdframe
\ifx\HCode\undefined
    \usepackage{fontspec}                       % Awesome font not available on html
    \usepackage{url}                            % Clickable URL's
    \hypersetup{hidelinks}                      % Hide links
\else
\fi

% Drawings
\usepackage[framemethod=TikZ]{mdframed}     % Box around text
\ifx\HCode\undefined
    \usepackage[absolute, overlay]{textpos}     % For Textblock
    \setlength{\TPHorizModule}{1cm}
    \setlength{\TPVertModule}{1cm}
\else
    \usepackage[overlay]{textpos}               % For Textblock
    \setlength{\TPHorizModule}{28pt}
    \setlength{\TPVertModule}{28pt}
\fi
\usepackage{graphicx}                       % Include foto
\usepackage{tabularx}                       % Coordinates
\usepackage{enumitem}                       % Aptitudes enumeration
\usepackage{genealogytree}                  % Born

% Display + Margin
\tolerance=1000


% Font
\usepackage{amsfonts}                       % Checkmark
\usepackage{relsize}                        % Scale font
\usepackage{everysel}
\EverySelectfont{%                          % This percent is important otherwise I got space befor paragph
    \fontdimen2\font=0.3em                  % Interword space
    \hyphenchar\font=`\-                    % Allow hyphenation
    \xspaceskip=0em
}


% Helper
\providecommand\mission[1]{\makebox[1.6cm][l]{\textbf{#1}}}
\let\cvitem\item


% ===========================================================================================================================


% -2/ must be configured
\newif\iftbftiny

\ifx\HCode\undefined
\else
    \usepackage{polyglossia}                    % Language
    \typeout{tbf : language \tbflanguage}
    \ifnum\pdfstrcmp{\tbflanguage}{french}=0
    \frenchspacing                              % Better looking spacings after periods
    \setmainlanguage{french}
    \fi
    \ifnum\pdfstrcmp{\tbflanguage}{english}=0
    \setmainlanguage{english}
    \fi
    \ifnum\pdfstrcmp{\tbflanguage}{spanish}=0
    \typeout{tbf : spanish}
    \setmainlanguage{spanish}
    \fi
\fi

% Language
\def\wordeducation{}
\def\wordwork{}
\def\wordskill{}
\def\wordsoftware{}
\def\wordlanguage{}
\ifnum\pdfstrcmp{\tbflanguage}{french}=0
\def\wordinfo{Informations personnels}
\def\wordeducation{Formation}
\def\wordwork{Expérience Professionnelle}
\def\wordskill{Compétence}
\def\wordsoftware{Logiciel}
\def\wordlanguage{Langue}
\fi
\ifnum\pdfstrcmp{\tbflanguage}{english}=0
\def\wordinfo{Personal Informations}
\def\wordeducation{Education}
\def\wordwork{Working Experience}
\def\wordskill{Skill}
\def\wordsoftware{Software}
\def\wordlanguage{Language}
\fi
\ifnum\pdfstrcmp{\tbflanguage}{spanish}=0
\def\wordinfo{Informaciones personales}
\def\wordeducation{Formación}
\def\wordwork{Experiencia Profesional}
\def\wordskill{Competencia}
\def\wordsoftware{Software}
\def\wordlanguage{Idioma}
\fi






% 1/ Commands
\newcommand\tbflogo[1]{\includegraphics[width=12pt, height=12pt]{#1}}

\newcommand\tbfimflag[1]{\includegraphics[height=12pt]{#1}\phantom{\tiny{1}}}

\newcommand\tbfsoft[2]{\includegraphics[height=12pt]{#2}\phantom{\tiny{1}}#1}

\newcommand\tbfpart[1]{#1}

\newcommand\tbfname[2]{
  \pgfmathsetmacro{\tbftmp}{\tbfbordertop + 0.6}
  \begin{textblock}{23}(0, \tbftmp)
     \centering{\headerfirstnamestyle{#1} \hspace{5pt} \headerlastnamestyle{#2}}
  \end{textblock}
}

\newcommand\tbffoto[1]{
    \ifx\HCode\undefined
        \begin{textblock}{23}(17.7, \tbfbordertop)
            \includegraphics[width=2.8cm]{#1}
        \end{textblock}
    \else
    \fi
}


\newcommand\tbfdescription[2][16pt]{
  \pgfmathsetmacro{\tbftmp}{\tbfbordertop + 1.7}
  \begin{textblock}{23}(0, \tbftmp)
  \begin{center}

  {\fontsize{#1}{0.6cm} \scshape\bfseries #2}

  \end{center}
  \end{textblock}
}

\newcommand\tbfskilltwo[1]{\section{#1}}


% 2/ Helpers
% Name, size, x, y, nada
\newsavebox{\internalbox}
\newenvironment*{internal}[5]
{
    \ifx\HCode\undefined
        \begin{textblock}{#2}(#3, #4)
        \begin{mdframed}
    \else
        \begin{lrbox}{\internalbox}
        \begin{minipage}[t]{0.5\textwidth}
    \fi

    \section{#1}
}
{

    \ifx\HCode\undefined
        \end{mdframed}
        \end{textblock}
    \else
        \end{minipage}
        \end{lrbox}
        \usebox{\internalbox}
    \fi

}

\newenvironment*{rounded_frame}[5]
{
    \mdfsetup{
      middlelinecolor=mybluegrey,
      middlelinewidth=1pt,
      backgroundcolor=white!0,
      roundcorner=10pt,
      skipabove=30,
      fontcolor=black,
      innertopmargin=-10pt,
      innerrightmargin=17pt,
    }
    \begin{internal}{#1}{#2}{#3}{#4}{#5}
}
{
    \end{internal}
}


\newenvironment{flat_frame}[5]
{
    \mdfsetup{
      middlelinecolor=mybluegrey,
      middlelinewidth=1pt,
      backgroundcolor=white!0,
      roundcorner=10pt,
      skipabove=30,
      fontcolor=black,
      innertopmargin=-10pt,
      innerrightmargin=17pt,
      middlelinecolor=white!0
    }
    \begin{internal}{#1}{#2}{#3}{#4}{#5}
}
{
    \end{internal}
}





% 3/ ENVIRONMENT



% [coordinate][Width, X, Y]
\newenvironmentx{coordinatelist}{
    \needspace{0.5\textheight}
    \newdimen\boxwidth
    \boxwidth=\dimexpr\linewidth-2\fboxsep\relax
    \pgfmathsetmacro{\tbftmpleft}{\tbfborderleft + 1}
    \begin{textblock}{7}(\tbfborderleft, \tbfbordertop)
    \begin{tabular}{l|l}
}{
    \end{tabular}
  \end{textblock}
}

% [job] { years }  {company } { more }
%                                Width, X (horizontal), Y (vertical)
\newenvironmentx{joblist}[3][1=13.2, 2=7.8, 3=3.9]{
  \begin{flat_frame}{\wordwork}{#1}{#2}{#3}{}
  \relscale{1.27}
  \renewcommand\item[4][]{
    \vspace*{0.3cm}
      \iftbftiny \setlength{\parskip}{-10pt} \fi
    {\raggedleft\textsc{\bodyfont{##2}}\\[1pt]}                 % YEAR

    \textsc\bfseries{\bodyfont{##1}}\\[1pt]                   % POSTE
    \textsc\bodyfont{##3}\\[1pt]                            % COMPANY
    \bodyfontlight\upshape{##4}                                 % DESCRIPTION
    }
}{
  \end{flat_frame}
}

% Formation
\newenvironmentx{yearlist}[3][1=7.1, 2=0.7, 3=3.9]{
  \begin{rounded_frame}{\wordeducation}{#1}{#2}{#3}{}
  \relscale{1.2}
    \renewcommand\item[4][]{
    \textsc{\bodyfont{##2}}                                     % YEAR
            \hspace*{0.2cm}                                         % TODO align them because letter 0 takes more space than 1
            \textsc\bfseries{\bodyfont{##1}} \\[1pt]                % DIPLOME
    \textsc\bodyfont{##3}   \\                          % ETABLISSEMENT
    \textit\descriptionstyle{##4} \\                            % DESCRIPTION
    \vspace{-0.3cm} % reduce horizontal space
    }
}{
  \end{rounded_frame}
}


% Compétence width, x, y
\newenvironmentx{skilllist}[3][1=7.1, 2=0.7, 3=15.2]{
  \begin{rounded_frame}{\wordlanguage}{#1}{#2}{#3}{}
  \relscale{1.2}
}{
  \end{rounded_frame}
}



\newcommand\tbfbirth[1]{
    \includegraphics[width=0.05\textwidth]{../Figure/logo/birthday1_128.png}    & \normalsize  \bodyfont{#1} \smallskip\\
}
\newcommand\tbfmail[1]{
    \Letter      & \normalsize  \bodyfont{#1}\smallskip\\
}
\newcommand\tbfaddr[1]{
    \Writinghand & \normalsize  \bodyfont{#1}\smallskip\\
}
\newcommand\tbftel[1]{
    \Telefon     & \normalsize  \bodyfont{#1}\smallskip\\
}
\newcommand\tbfvoidpersonal[1]{
    \hspace{1mm}    & \normalsize  \bodyfont{#1}\smallskip\\
}
\newcommand\tbfflag[1]{
    \includegraphics[width=0.05\textwidth]{../Figure/flag.png}    & \normalsize  \bodyfont{#1} \smallskip\\
}
\newcommand\tbfcar[1]{
    \includegraphics[width=0.05\textwidth]{../Figure/logo/car1_128.png}    & \normalsize  \bodyfont{#1} \smallskip\\
}
\newcommand\tbfwww[1]{
    \includegraphics[width=0.05\textwidth]{../Figure/logo/www_128.png}    & \normalsize  \bodyfont{#1} \smallskip\\
}


\newenvironment{languagelist}{
  \renewcommand\item[2]{
    ##1 & ##2 \\
  }
  \begin{tabular}[t]{ll}
}{
  \end{tabular}
}

\newenvironment{stufflist}{
  \renewcommand\item[2]{
    ##1 & ##2 \\
  }
  \setlength\tabcolsep{1.5pt} % default value: 6pt
  \begin{tabular}[]{p{0.1cm}p{6cm}}
}{
  \end{tabular}
}


\documentclass[a4paper,12pt]{article}
% Default settings
\providecommand{\tbfcmd}{professional}

\usepackage[utf8]{inputenc}
\usepackage{xargs}                              % Key - value parameters newenvironmentx
\usepackage{marvosym}
\usepackage{newunicodechar}

\usepackage[overlay,absolute]{textpos}              % For Textblock
  \setlength{\TPHorizModule}{1cm}
  \setlength{\TPVertModule}{1cm}
\usepackage{relsize}                        % Scale font (for relscale)

\usepackage[a4paper,bindingoffset=0.2in,%
            left=-1cm,right=1cm,top=1cm,bottom=1cm,%
            footskip=.25in]{geometry}
    \setlength{\textwidth}{21cm}


%A Few Useful Packages
\usepackage{fontspec}                     %for loading fonts
\usepackage{xunicode,xltxtra,url,parskip}     %other packages for formatting
\RequirePackage{color,graphicx}
\usepackage[usenames,dvipsnames]{xcolor}
\usepackage[big]{layaureo}                 %better formatting of the A4 page
% an alternative to Layaureo can be ** \usepackage{fullpage} **
\usepackage{supertabular}                 %for Grades
\usepackage{titlesec}                    %custom \section

%Setup hyperref package, and colours for links
\usepackage{hyperref}
\definecolor{linkcolour}{rgb}{0,0.2,0.6}
\hypersetup{colorlinks,breaklinks,urlcolor=linkcolour, linkcolor=linkcolour}

%FONTS
\defaultfontfeatures{Mapping=tex-text}
%\setmainfont[SmallCapsFont = Fontin SmallCaps]{Fontin}
%%% modified for Karol Kozioł for ShareLaTeX use
\setmainfont[
Path = ../fonts/,
SmallCapsFont = Fontin-SmallCaps.otf,
BoldFont = Fontin-Bold.otf,
ItalicFont = Fontin-Italic.otf
]
{Fontin.otf}
%%%

%CV Sections inspired by: 
%http://stefano.italians.nl/archives/26
\titleformat{\section}{\Large\scshape\raggedright}{}{0em}{}[\titlerule]
\titlespacing{\section}{0pt}{3pt}{3pt}
%Tweak a bit the top margin
\addtolength{\voffset}{-1.3cm}
\addtolength{\hoffset}{-1.3cm}

%Italian hyphenation for the word: ''corporations''
\hyphenation{im-pre-se}

\setlength{\parindent}{0pt}


% Now I add 
\usepackage{amsfonts}                                             % For checkmark
\usepackage{longtable}                                            % Spread table on multiple pages






\ifx\HCode\undefined
\else
    \usepackage{polyglossia}                    % Language
    \typeout{tbf : language \tbflanguage}
    \ifnum\pdfstrcmp{\tbflanguage}{french}=0
    \frenchspacing                              % Better looking spacings after periods
    \setmainlanguage{french}
    \fi
    \ifnum\pdfstrcmp{\tbflanguage}{english}=0
    \setmainlanguage{english}
    \fi
    \ifnum\pdfstrcmp{\tbflanguage}{spanish}=0
    \typeout{tbf : spanish}
    \setmainlanguage{spanish}
    \fi
\fi

% Language
\def\wordeducation{}
\def\wordwork{}
\def\wordskill{}
\def\wordsoftware{}
\def\wordlanguage{}
\ifnum\pdfstrcmp{\tbflanguage}{french}=0
\def\wordinfo{Informations personnels}
\def\wordeducation{Formation}
\def\wordwork{Expérience Professionnelle}
\def\wordskill{Compétence}
\def\wordsoftware{Logiciel}
\def\wordlanguage{Langue}
\fi
\ifnum\pdfstrcmp{\tbflanguage}{english}=0
\def\wordinfo{Personal Informations}
\def\wordeducation{Education}
\def\wordwork{Working Experience}
\def\wordskill{Skill}
\def\wordsoftware{Software}
\def\wordlanguage{Language}
\fi
\ifnum\pdfstrcmp{\tbflanguage}{spanish}=0
\def\wordinfo{Informaciones personales}
\def\wordeducation{Formación}
\def\wordwork{Experiencia Profesional}
\def\wordskill{Competencia}
\def\wordsoftware{Software}
\def\wordlanguage{Idioma}
\fi


%End of include

% Fix
\providecommand{\pgfsyspdfmark}[3]{}


\newif\iftbftiny


\pagestyle{empty} % non-numbered pages
\font\fb=''[cmr10]'' %for use with \LaTeX command



\let\cvitem\item
\providecommand\mission[1]{\makebox[1.6cm][l]{\textbf{#1}}}


%%
%        Name
%% 
\newcommand{\tbfname}[2]{
    \par{\centering{
        \Huge{#1 \textsc{#2}}
    }\bigskip\par}
}

\newcommand\tbffoto[1]{
    \ifx\HCode\undefined
        \begin{textblock}{23}(17.7, 1.5)
            \includegraphics[width=2.8cm]{#1}
        \end{textblock}
    \else
    \fi
}

\newcommand{\tbfdescription}[2][16pt]{
  \begin{center}
  {\fontsize{#1}{0.6cm} \scshape\bfseries #2}
  \end{center}
}


\newenvironmentx{yearlist}[3][1=0, 2=0, 3=0]{
    \renewcommand\item[4][]{
        ##2 & \large\textbf{##1}                \\*
        & \normalsize\textbf{##3}               \\*
        & \begin{minipage}[b]{0.8\textwidth}
            ##4
          \end{minipage}                        \\*
        \multicolumn{2}{c}{}                    \\
    }
    \section{\wordeducation}
    \begin{longtable}[l]{@{}p{.20\textwidth} p{.80\textwidth}} 
}{
    \end{longtable}
}


\newenvironmentx{joblist}[3][1=0, 2=0, 3=0]{
    \renewcommand\item[4][]{
        ##2 & \large\textbf{##1}                \\*
        & \emph{##3}                            \\*
        & \begin{minipage}[b]{0.8\textwidth}
           ##4
          \end{minipage}                        \\*
        \multicolumn{2}{c}{}                    \\
    }
    % Note: tabular must be after the defition
    \section{\wordwork}
    \begin{longtable}[l]{@{}p{.20\textwidth} p{.80\textwidth}} 
}{
    \end{longtable}
}


\newenvironmentx{skilllist}[3][1=0, 2=0, 3=0]{}


\newenvironment{languagelist}{
    \begin{samepage}
    \section{\wordlanguage}
    \renewcommand\item[2]{
    ##1 & ##2 \\
    }
    % Note: tabular must be after the defition
    % Note: The @{} is to remove indentation
    \begin{longtable}[l]{@{}p{.20\textwidth} p{.80\textwidth}} 
}{
    \end{longtable}
    \end{samepage}
}


\newenvironment{coordinatelist}{
    \section{\wordinfo}
    \begin{tabular}{rl}
}{
    \end{tabular}
}

\newcommand{\tbfskilltwo}[2]{
    \section{#1}
    #2
}



% 0/ Go
\begin{document}


% 0.1/ Header
\tbfname{\vspace{0.3cm}Jaime}{Sanhueza}


\tbfdescription[20pt]{
    \vspace{0.4cm}
    Consulor,
    Ingeniero de proyectos
}

\tbffoto{../Figure/jaime_m1.jpg}

\begin{coordinatelist}
  \tbfmail{jaime.sanhueza-wilson@hotmail.com}
  \tbfbirth{8 de Junio 1963}
  \tbftel{+56 9 88 14 33 19 }
  \tbfaddr{Santiago, Chile}
\end{coordinatelist}





% 2/ Job
\begin{joblist}[12.8][8.4][4]


\item[Soporte y entrenamiento]{2019}
  {
  \href{https://www.enex.cl/}{Enex}, Santiago, Chile}
  {
    Soporte y Capacitación a Distribuidores en la operación de EDS Plataforma Automatización Orpak (Estaciones DO y RBA) ; Puesta en marcha y capacitación sistema autoservicio gasolinas Easy Pay
  }
  

\item[Analista de soporte]{2018 -- 2019}
  {
  \href{https://www.movistar.cl/}{Movistart}, Santiago, Chile}
  {
    Encargado de identificar, diagnosticar y gestionar la resolución de incidentes o problemas, asegurando la continuidad operacional de los sistemas informáticos del cliente en las plataformas de telefonía celular de acuerdo a las definiciones del servicio establecido.

    Vela para que los diversos roles del modelo de operación cumplan con sus funciones y actividades; Efectúa supervisión general, controla y gestiona los principales indicadores y KPI de los servicios críticos con el fin de actuar proactivamente ante riesgos.
    Monitoreo y control de Servicios; Coordina y asegura la resolución de incidencias Críticas y Altas y reporta el estado de avance de la solución, como también asegura la generación de los RCA asociados a las incidencias antes mencionadas; Controla la gestión de requerimientos especiales sobre el servicio y la generación de reportes de control cuantitativos y cualitativos de los mismos; Seguimiento y control para la correcta ejecución de los diferentes ciclos comerciales (facturación, EOD, EOC, EOM)
  }

  
\item[Consultor en soporte de proyectos]{2016 -- 2018}
  {
  \href{https://www.enex.cl/}{Enex}, Santiago, Chile}
  {
    Responsable del Roll Out en proyecto Host to Host de integración tecnológica de solución Transbank y plataforma Orpak (sistema de control y automatización de venta y despacho de combustible) en un sistema seguro asociado a la compra con tarjetas en las EdS de la red Shell (combustibles y tiendas de conveniencia).

    \checkmark  Migraciones efectivas a más de 95 tiendas de conveniencia tanto de operación directa Enex y concesionadas. La coordinación con el cliente interno y externo, junto a una dedicada supervisión y el conocimiento del negocio permitió la implementación exitosa del sistema en las eds.

    \checkmark  Pruebas de funcionamiento transaccionales en pos de venta y entrenamiento a la dotación de cajeros y administradores.

    \checkmark  Supervisión y coordinación del soporte de terreno reactivo de fallas (mesa de ayuda Orpak)

    \checkmark  Implementación de estaciones piloto del proyecto Host to Host en combustibles
  }


\item[Analista de soporte comercial]{2016}
  {
  \href{https://www.enex.cl/}{Enex}, Santiago, Chile}
  {
    Soporte en migraciones a la solución de facturación electrónica (Paperless) en eds Shell de operación directa y concesionadas para la generación de la documentación electrónica, integrándose con los sistemas de gestión de los distribuidores en las estaciones de servicio.

    \checkmark  Control y seguimiento de proceso de migraciones y coordinación con distribuidores y el soporte técnico (Orpak/Paperless) para la actividad de migración exitosa en la forma establecida por el negocio
  }


\item[Consultor en soporte de proyectos]{2014 -- 2016}
  {
  \href{https://www.enex.cl/}{Enex}, Santiago, Chile}
  {
    Soporte migraciones proyecto Host to Host en tiendas de conveniencia de la red Shell (operación directa y concesionarios)

    Implementación de nuevo medio de pago tarjeta SCE (Tarjeta prepago Shellcard flota) y Shellcard Taxi. Tarjeta de prepago para la compra de combustibles en la red de estaciones de servicio Shell asociadas al sistema de control Orpak.

    \checkmark   Habilitaciones exitosas en más de 250 estaciones de servicio, incluyendo pruebas transaccionales y entrenamiento a la dotación de atendedores, jefes de playa y administradores
  }

\item[Soporte proyecto]{2010 -- 2014}
  {
  \href{https://www.shell.com/}{Shell}, Santiago, Chile}
  {
    Proyecto migración tarjeta Shellcard. Roll out de integración de plataforma tarjeta Shellcard (venta crédito diésel transportistas) a sistema Pumplink-Ges. Proyecto Pumplink-Transactor (automatización y control de venta petróleo diesel transportistas en puntos de venta industriales y concesionadas).

    \checkmark Migraciones exitosas en terminales de venta (pos) en modalidad standalone a controlada (pumplink); Sobre 80 estaciones de servicio de la red Shell

    \checkmark Implementaciones plataforma Pumplink-Transactor; Sistema de automatización para el despacho y control de combustible diesel integrado a surtidores en clientes industriales y concesionados (22 puntos). Licitaciones y administración en la ejecución de obras civiles en nuevas instalaciones (definición y control de presupuestos, actividades y obras)
  }

\item[Analista Back-Office Shellcard (Soporte comercial)]{2007 -- 2009}
  {
  \href{https://www.shell.com/}{Shell}, Santiago, Chile}
  {
     Analista backoffice tarjeta shellcard (Tarjeta de crédito para venta de diesel segmento Transportes); Facturación shellcard y procesos de cierre ventas quincenales; Administración de tarjeta Shellcard, contratos clientes, emisión, activaciones, administración de plataforma web shellcard
  }

\item[Analista Soporte IT Retail]{1996 -- 2006}
  {
  \href{https://www.shell.com/}{Shell}, Santiago, Chile}
  {
    Analista IT Retail Automation-Network Support

    \checkmark    Análisis y soporte On Site en proyecto de sistema GES Opc System-RBA (Gestión Estaciones Servicio-Retail Busines Allied). Años 2004-2006. Levantamiento de requerimientos de funcionalidad de sistema y procedimientos de pruebas.

    \checkmark    Coordinador para la gestión de incidentes de infraestructura y aplicaciones y focal point de soporte para la mesa de ayuda (servicio externo) y puntos de venta. Soporte remoto.

    \checkmark   Responsable en la administración de stock, administración y control de adquisiciones del proyecto. Administración de compras de hardware (3 proveedores) durante el proyecto y necesidades futuras.

    \checkmark    Responsable en la implementación de plataforma de gestión en segmento retail. Instalación de nueva plataforma de punto de venta en tiendas de conveniencia de la red Shell (Select, Express market) de cobertura nacional

    \checkmark    Apoyo a Backoffice shellcard; Comunicación y transferencia de datos desde servidores de comunicación (PEC Central) y sites remotos de la red (80 estaciones de servicio Shellcard a nivel nacional). Control de la actividad y validación de la información asegurando la correcta y completa ejecución de la misma para el negocio.

    \checkmark    Coordinador del soporte técnico y mesa de ayuda externo a nivel nacional (COASIN) para la solución de incidentes de la plataforma Shellcard (HD y SW instalados). Definición de contratos de soporte para la continuidad operativa (tipificación y tratamiento de incidentes, SLA’s).
  }


\item[Analista de programación]{1995}
  {
  \href{https://www.gerdau.cl/}{Gerdau}, Santiago, Chile}
  {
     Analista backoffice tarjeta shellcard (Tarjeta de crédito para venta de diesel segmento Transportes); Facturación shellcard y procesos de cierre ventas quincenales; Administración de tarjeta Shellcard, contratos clientes, emisión, activaciones, administración de plataforma web shellcard
  }


\item[Supervisor de operaciones]{1992 -- 1995}
  {
  \href{https://www.bellsouth.com}{Bellsouth}, Santiago, Chile}
  {
    Supervisión sala de cómputos. Coordinación y control de la dotación de 4 operadores y explotación de aplicaciones celulares.

     \checkmark   Responsabilidad en la gestión y control de sistema de turnos (planeamiento, asignación y control de turnos flexibles en la operación y explotación de sala). Esto conllevó a un aumento de un 20\% (promedio) en la productividad y en la reducción de los tiempos de ejecución de procesos sistemáticos (facturaciones semanales de tráfico celular de clientes) y una mejora en la percepción del servicio del cliente interno (Operadoras de turno de servicio a cliente).

     \checkmark   Operación de plataformas computacionales (IBM y VAX-VMS) y administración básica de servidor de datos Informix on-line. Procesamiento y mantención de datos (backup, gestión de cuentas y perfiles de usuario).
  }


\end{joblist}





% 1/ Education
\begin{yearlist}[7.7][\tbfborderleft][4]

\relscale{1.1}
\tolerance=100

\item[Informático]{1987 -- 1990}
  { % TODO get logo
  % \tbflogo{../Figure/logo32/puc_32.png}
  \href{http://www.duoc.cl/}{Duoc UC}, Santiago, Chile
  }
  {   Análisis de sistemas informáticos}

\end{yearlist}



% 3/ Skill
\begin{skilllist}[7.7][\tbfborderleft][16.8]

\relscale{1.1}
\tolerance=9999
\emergencystretch 0.1cm

\vspace{0.2cm}
\begin{languagelist}
\item{\tbfimflag{../Figure/flag/spain_32x85.png}Español}     {idioma materno}
\item{\tbfimflag{../Figure/flag/great_britain_32x85.png}Inglés}     {oral y escrito}
\end{languagelist}
\vspace{-0.3cm}


\tbfskilltwo{Habilidades}{
    \item[\textbullet]{Implementación y Gestión de Proyectos}
    \item[\textbullet]{Metodología de Gestión de Proyectos}
    \item[\textbullet]{Proyectos de Tecnología}
}

\end{skilllist}




\end{document}
