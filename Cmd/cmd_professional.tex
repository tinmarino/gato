\documentclass[a4paper,12pt]{article}
% Default settings
\providecommand{\tbfcmd}{professional}

\usepackage[utf8]{inputenc}
\usepackage{xargs}                              % Key - value parameters newenvironmentx
\usepackage{marvosym}
\usepackage{newunicodechar}

\usepackage[overlay,absolute]{textpos}              % For Textblock
  \setlength{\TPHorizModule}{1cm}
  \setlength{\TPVertModule}{1cm}
\usepackage{relsize}                        % Scale font (for relscale)

\usepackage[a4paper,bindingoffset=0.2in,%
            left=-1cm,right=1cm,top=1cm,bottom=1cm,%
            footskip=.25in]{geometry}
    \setlength{\textwidth}{21cm}


%A Few Useful Packages
\usepackage{fontspec}                     %for loading fonts
\usepackage{xunicode,xltxtra,url,parskip}     %other packages for formatting
\RequirePackage{color,graphicx}
\usepackage[usenames,dvipsnames]{xcolor}
\usepackage[big]{layaureo}                 %better formatting of the A4 page
% an alternative to Layaureo can be ** \usepackage{fullpage} **
\usepackage{supertabular}                 %for Grades
\usepackage{titlesec}                    %custom \section

%Setup hyperref package, and colours for links
\usepackage{hyperref}
\definecolor{linkcolour}{rgb}{0,0.2,0.6}
\hypersetup{colorlinks,breaklinks,urlcolor=linkcolour, linkcolor=linkcolour}

%FONTS
\defaultfontfeatures{Mapping=tex-text}
%\setmainfont[SmallCapsFont = Fontin SmallCaps]{Fontin}
%%% modified for Karol Kozioł for ShareLaTeX use
\setmainfont[
Path = ../fonts/,
SmallCapsFont = Fontin-SmallCaps.otf,
BoldFont = Fontin-Bold.otf,
ItalicFont = Fontin-Italic.otf
]
{Fontin.otf}
%%%

%CV Sections inspired by: 
%http://stefano.italians.nl/archives/26
\titleformat{\section}{\Large\scshape\raggedright}{}{0em}{}[\titlerule]
\titlespacing{\section}{0pt}{3pt}{3pt}
%Tweak a bit the top margin
\addtolength{\voffset}{-1.3cm}
\addtolength{\hoffset}{-1.3cm}

%Italian hyphenation for the word: ''corporations''
\hyphenation{im-pre-se}

\setlength{\parindent}{0pt}


% Now I add 
\usepackage{amsfonts}                                             % For checkmark
\usepackage{longtable}                                            % Spread table on multiple pages






\ifx\HCode\undefined
\else
    \usepackage{polyglossia}                    % Language
    \typeout{tbf : language \tbflanguage}
    \ifnum\pdfstrcmp{\tbflanguage}{french}=0
    \frenchspacing                              % Better looking spacings after periods
    \setmainlanguage{french}
    \fi
    \ifnum\pdfstrcmp{\tbflanguage}{english}=0
    \setmainlanguage{english}
    \fi
    \ifnum\pdfstrcmp{\tbflanguage}{spanish}=0
    \typeout{tbf : spanish}
    \setmainlanguage{spanish}
    \fi
\fi

% Language
\def\wordeducation{}
\def\wordwork{}
\def\wordskill{}
\def\wordsoftware{}
\def\wordlanguage{}
\ifnum\pdfstrcmp{\tbflanguage}{french}=0
\def\wordinfo{Informations personnels}
\def\wordeducation{Formation}
\def\wordwork{Expérience Professionnelle}
\def\wordskill{Compétence}
\def\wordsoftware{Logiciel}
\def\wordlanguage{Langue}
\fi
\ifnum\pdfstrcmp{\tbflanguage}{english}=0
\def\wordinfo{Personal Informations}
\def\wordeducation{Education}
\def\wordwork{Working Experience}
\def\wordskill{Skill}
\def\wordsoftware{Software}
\def\wordlanguage{Language}
\fi
\ifnum\pdfstrcmp{\tbflanguage}{spanish}=0
\def\wordinfo{Informaciones personales}
\def\wordeducation{Formación}
\def\wordwork{Experiencia Profesional}
\def\wordskill{Competencia}
\def\wordsoftware{Software}
\def\wordlanguage{Idioma}
\fi


%End of include

% Fix
\providecommand{\pgfsyspdfmark}[3]{}


\newif\iftbftiny


\pagestyle{empty} % non-numbered pages
\font\fb=''[cmr10]'' %for use with \LaTeX command



\let\cvitem\item
\providecommand\mission[1]{\makebox[1.6cm][l]{\textbf{#1}}}


%%
%        Name
%% 
\newcommand{\tbfname}[2]{
    \par{\centering{
        \Huge{#1 \textsc{#2}}
    }\bigskip\par}
}

\newcommand\tbffoto[1]{
    \ifx\HCode\undefined
        \begin{textblock}{23}(17.7, 1.5)
            \includegraphics[width=2.8cm]{#1}
        \end{textblock}
    \else
    \fi
}

\newcommand{\tbfdescription}[2][16pt]{
  \begin{center}
  {\fontsize{#1}{0.6cm} \scshape\bfseries #2}
  \end{center}
}


\newenvironmentx{yearlist}[3][1=0, 2=0, 3=0]{
    \renewcommand\item[4][]{
        ##2 & \large\textbf{##1}                \\*
        & \normalsize\textbf{##3}               \\*
        & \begin{minipage}[b]{0.8\textwidth}
            ##4
          \end{minipage}                        \\*
        \multicolumn{2}{c}{}                    \\
    }
    \section{\wordeducation}
    \begin{longtable}[l]{@{}p{.20\textwidth} p{.80\textwidth}} 
}{
    \end{longtable}
}


\newenvironmentx{joblist}[3][1=0, 2=0, 3=0]{
    \renewcommand\item[4][]{
        ##2 & \large\textbf{##1}                \\*
        & \emph{##3}                            \\*
        & \begin{minipage}[b]{0.8\textwidth}
           ##4
          \end{minipage}                        \\*
        \multicolumn{2}{c}{}                    \\
    }
    % Note: tabular must be after the defition
    \section{\wordwork}
    \begin{longtable}[l]{@{}p{.20\textwidth} p{.80\textwidth}} 
}{
    \end{longtable}
}


\newenvironmentx{skilllist}[3][1=0, 2=0, 3=0]{}


\newenvironment{languagelist}{
    \begin{samepage}
    \section{\wordlanguage}
    \renewcommand\item[2]{
    ##1 & ##2 \\
    }
    % Note: tabular must be after the defition
    % Note: The @{} is to remove indentation
    \begin{longtable}[l]{@{}p{.20\textwidth} p{.80\textwidth}} 
}{
    \end{longtable}
    \end{samepage}
}


\newenvironment{coordinatelist}{
    \section{\wordinfo}
    \begin{tabular}{rl}
}{
    \end{tabular}
}

\newcommand{\tbfskilltwo}[2]{
    \section{#1}
    #2
}
