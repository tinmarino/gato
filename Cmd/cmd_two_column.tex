\documentclass[paper=a4, fontsize=11pt]{tccv}          % Unfortunately must keep some stuff in class for now



% 0/ Default settings
\providecommand{\tbfbordertop}{0.6}
\providecommand{\tbfborderleft}{0.6}
\providecommand{\tbflanguage}{english}
\providecommand{\tbfcmd}{two_column}

% API
\usepackage{xargs}                          % Key - value parameters
\usepackage{xcolor}
\usepackage{pgf}                            % Math macro
\usepackage{xifthen}                        % If, else block

% Drawings
\usepackage[framemethod=TikZ]{mdframed}     % Box around text
\ifx\HCode\undefined
    \usepackage[absolute, overlay]{textpos}     % Textblock
    \setlength{\TPHorizModule}{1cm}
    \setlength{\TPVertModule}{1cm}
\else
    % No package for tex4ht
\fi
\usepackage{graphicx}                       % Include foto
\usepackage{tabularx}                       % Coordinates
\usepackage{enumitem}                       % Aptitudes enumeration
\usepackage{genealogytree}                  % Born

% Display + Margin
\tolerance=1000


% Font
\usepackage{amsfonts}                       % Checkmark
\usepackage{relsize}                        % Scale font
\usepackage{everysel}
\EverySelectfont{%                          % This percent is important otherwise I got space befor paragph
    \fontdimen2\font=0.3em                  % Interword space
    \hyphenchar\font=`\-                    % Allow hyphenation
    \xspaceskip=0em
}


% Named arguments
\usepackage{keyval,xparse}% http://ctan.org/pkg/{keyval,xparse}
\makeatletter
% ========= KEY DEFINITIONS =========
\define@key{tbfarg}{width}{\def\mmwidth{#1}}
\define@key{tbfarg}{x}{\def\mmx{#1}}
\define@key{tbfarg}{y}{\def\mmy{#1}}
\define@key{tbfarg}{fontsize}{\def\mmfontsize{#1}}
%\define@key{tbfarg}{fonsize}{\def\mmfontsize{#1}}
\makeatother


% Helper
\providecommand\mission[1]{\makebox[1.6cm][l]{\textbf{#1}}}
\let\cvitem\item

% Helper : tbfblock : textblock for html
% from https://tex.stackexchange.com/questions/185933/is-there-an-alternative-to-textpos-when-using-pgfpages
\makeatletter
\pgfmathsetmacro{\htmlfactor}{2.3} % lager in html
\ifx\HCode\undefined
    \def\@tbfblock#1(#2, #3){\begin{textblock}{#1}(#2, #3)}
    \def\@endtbfblock{\end{textblock}}
    \newcommand\tbfvspace[1]{\vspace*{#1}}
    \newcommand\tbfhspace[1]{\hspace*{#1}}
\else
    \def\@tbfblock#1(#2, #3){
        \pgfmathsetmacro{\tmpwidth}{int(#1 * 28.45 * \htmlfactor)}
        \pgfmathsetmacro{\tmpleft}{int(#2 * 28.45 * \htmlfactor)}
        \pgfmathsetmacro{\tmptop}{int(#3 * 28.45 * \htmlfactor)}
        \HCode{<div class="textblock"
            style="
                float: left;
                position: absolute;
                % display: inline-block;
                width: \tmpwidth px;
                margin-left: \tmpleft px;
                margin-top: \tmptop px;
            "
        >\Hnewline}
        %\begin{minipage}[]{ \tbftmp\textwidth }
        % \begin{flushleft}
    }
    \def\@endtbfblock{\HCode{</div>\Hnewline}}


    % Custom html Block with custom attribute (i.e. style)
    \def\@tbfhtmlblock#1(#2, #3, #4){
        \def\htmlstyle{#4}
        \HCode{<div class="textcustomblock"
            \htmlstyle
        >\Hnewline}
    }
    \expandafter\let\csname tbfhtmlblock\endcsname\@tbfhtmlblock
    \def\@endtbfhtmlblock{\HCode{</div>\Hnewline}}
    \expandafter\let\csname endtbfhtmlblock\endcsname\@endtbfhtmlblock

    \newcommand\tbfvspace[1]{
        \pgfmathsetmacro{\tmpmargintop}{int(#1 * \htmlfactor)}
        \HCode{<br style="display: block; margin-top: \tmpmargintop px;">}
    }
    \newcommand\tbfhspace[1]{
        \pgfmathsetmacro{\tmpmarginleft}{int(#1 * \htmlfactor)}
        \HtmlParOff
        \HCode{<span class="hspace" style="
            display:inline-block;
            margin-right: \tmpmarginleft px;"></span>
        }
    }
\fi
\expandafter\let\csname tbfblock\endcsname\@tbfblock
\expandafter\let\csname endtbfblock\endcsname\@endtbfblock
\makeatother

% ===========================================================================================================================


% 0.1/ must be configured
\newif\iftbftiny

\ifx\HCode\undefined
\else
    \usepackage{polyglossia}                    % Language
    \typeout{tbf : language \tbflanguage}
    \ifnum\pdfstrcmp{\tbflanguage}{french}=0
    \frenchspacing                              % Better looking spacings after periods
    \setmainlanguage{french}
    \fi
    \ifnum\pdfstrcmp{\tbflanguage}{english}=0
    \setmainlanguage{english}
    \fi
    \ifnum\pdfstrcmp{\tbflanguage}{spanish}=0
    \typeout{tbf : spanish}
    \setmainlanguage{spanish}
    \fi
\fi

% Language
\def\wordeducation{}
\def\wordwork{}
\def\wordskill{}
\def\wordsoftware{}
\def\wordlanguage{}
\ifnum\pdfstrcmp{\tbflanguage}{french}=0
\def\wordinfo{Informations personnels}
\def\wordeducation{Formation}
\def\wordwork{Expérience Professionnelle}
\def\wordskill{Compétence}
\def\wordsoftware{Logiciel}
\def\wordlanguage{Langue}
\fi
\ifnum\pdfstrcmp{\tbflanguage}{english}=0
\def\wordinfo{Personal Informations}
\def\wordeducation{Education}
\def\wordwork{Working Experience}
\def\wordskill{Skill}
\def\wordsoftware{Software}
\def\wordlanguage{Language}
\fi
\ifnum\pdfstrcmp{\tbflanguage}{spanish}=0
\def\wordinfo{Informaciones personales}
\def\wordeducation{Formación}
\def\wordwork{Experiencia Profesional}
\def\wordskill{Competencia}
\def\wordsoftware{Software}
\def\wordlanguage{Idioma}
\fi




% 1/ Commands
\newcommand\tbfskilltwo[1]{\section{#1}}

% Test
\NewDocumentCommand{\tbftest}{m}{%
    % https://stackoverflow.com/questions/1812214/latex-optional-arguments
    %\begingroup%
    % ========= KEY DEFAULTS + new ones =========
    \setkeys{tbfarg}{x={x default arg},y={y defaults},width={width default},#1}%
    toto
    x arg: \mmx \par
    y arg: \mmy \par
    %\endgroup%
}

% Name
\NewDocumentCommand{\tbfname}{O{} m m}{%
    \setkeys{tbfarg}{x={0},y={\tbfbordertop},width={23},#1}
    \begin{tbfblock}{\mmwidth}(\mmx, \mmy)
    \begin{center}
    \ifx\HCode\undefined
        \headerfirstnamestyle{#2}
    \else
        \HCode{<span style="
            font-size: 319\%;
            font-color: \#3A3A3A;
            font-weight: lighter;
        ">}
        #2
        \HCode{</span>}
    \fi
        \tbfhspace{5pt}
        \headerlastnamestyle{#3}
    \end{center}
    \end{tbfblock}
}

% Foto
\NewDocumentCommand{\tbffoto}{O{} m}{%
    \setkeys{tbfarg}{x={17.7},y={\tbfbordertop},width={23},#1}
    \ifx\HCode\undefined
        \begin{tbfblock}{\mmwidth}(\mmx, \mmy)
            \includegraphics[width=2.8cm]{#2}
        \end{tbfblock}
    \else
        % Do not include foto in html
    \fi
}


% Decrition
%\newcommand\tbfdescription[2][16pt]{
\NewDocumentCommand{\tbfdescription}{O{} m}{%
    \pgfmathsetmacro{\tbftmp}{\tbfbordertop + 1.7}
    \setkeys{tbfarg}{fontsize={16pt},x={0},y={\tbftmp},width={23},#1}
    %\newcommand\tbfdescription[2][16pt]{
    \begin{tbfblock}{\mmwidth}(\mmx, \mmy)
    \begin{center}

        {\fontsize{\mmfontsize}{17pt} \textbf{\textsc{#2}}}

    \end{center}
    \end{tbfblock}
}


% Even with floating environment
\newcommand\tbfpage{
    \clearpage
    \newpage
    \mbox{~}
    \clearpage
    \newpage
}

\newcommand*{\tbfstyledegree}[1]{
    \ifx\HCode\undefined
        \bodyfont{#1}
    \else
        \textbf{\textsc{\bodyfont{#1}}}
    \fi
}
\newcommand*{\tbfstyleinstitution}[1]{
    \ifx\HCode\undefined
        \bodyfontlight{#1}
    \else
        \textsc{\bodyfont{#1}}
    \fi
}



%%%%%%%%%%%%%%%%
% 3/ ENVIRONMENT
%%%%%%%%%%%%%%%%



% Helper: internal to position 
% Args: Name, size, x, y, nada
\newsavebox{\internalbox}
\newenvironment*{internal}[5]
{
    \ifx\HCode\undefined
        \begin{tbfblock}{#2}(#3, #4)
        \begin{mdframed}
        \section{#1}
    \else
        \ifx&#5&%
            \pgfmathsetmacro{\tmpwidth}{int(#2 * 28.45 * \htmlfactor)}
            \pgfmathsetmacro{\tmpleft}{int(#3 * 28.45 * \htmlfactor)}
            \pgfmathsetmacro{\tmptop}{int(#4 * 28.45 * \htmlfactor)}
            \def\tmparg{style="
                float: left;
                position: absolute;
                width: \tmpwidth px;
                margin-left: \tmpleft px;
                margin-top: \tmptop px;
            "}
        \else
            \def\tmparg{#5}
        \fi
        \begin{tbfhtmlblock}{#2}(#3, #4, \tmparg)

        % Font size 132 as section
        % \section{#1} No because the bar is ugly : I must define it TODO
        \HCode{<h3 style="font-size: 132\%;"> #1</h3>}
    \fi

}
{

    \ifx\HCode\undefined
        \end{mdframed}
        \end{tbfblock}
    \else
        \end{tbfhtmlblock}
    \fi
}

% Helper: frame without border
\newenvironment{flat_frame}[5]
{
    \mdfsetup{
      middlelinecolor=mybluegrey,
      middlelinewidth=1pt,
      backgroundcolor=white!0,
      roundcorner=10pt,
      skipabove=30,
      fontcolor=black,
      innertopmargin=-10pt,
      innerrightmargin=17pt,
      middlelinecolor=white!0
    }
    \begin{internal}{#1}{#2}{#3}{#4}{#5}
}
{
    \end{internal}
}

% Helper: frame with border
\newenvironment*{rounded_frame}[5]
{
    \mdfsetup{
      middlelinecolor=mybluegrey,
      middlelinewidth=1pt,
      backgroundcolor=white!0,
      roundcorner=10pt,
      skipabove=30,
      fontcolor=black,
      innertopmargin=-10pt,
      innerrightmargin=17pt,
    }
    \begin{internal}{#1}{#2}{#3}{#4}{#5}
}
{
    \end{internal}
}


% Coordinates
\NewDocumentEnvironment{coordinatelist}{O{}}{%
    \setkeys{tbfarg}{x={\tbfborderleft},y={\tbfbordertop},width={7},#1}
    \needspace{0.5\textheight}
    \newdimen\boxwidth
    \boxwidth=\dimexpr\linewidth-2\fboxsep\relax
    \begin{tbfblock}{\mmwidth}(\mmx, \mmy)
    \begin{tabular}{l|l}
}{
    \end{tabular}
    \end{tbfblock}
}

\newcommand\ralign[1]{\begin{center} #1 \end{center}}


% Jobs { years }  {company } { more }
% Width, X (horizontal), Y (vertical)
\newenvironmentx{joblist}[3][1=13.2, 2=7.8, 3=3.9]{
    \pgfmathsetmacro{\tmpwidth}{int(#1 * 28.45 * \htmlfactor)}
    \pgfmathsetmacro{\tmpleft}{int(#2 * 28.45 * \htmlfactor)}
    \pgfmathsetmacro{\tmptop}{int(#3 * 28.45 * \htmlfactor)}
    \begin{flat_frame}{\wordwork}{#1}{#2}{#3}{
        style="
            display: inline-block;
            max-width: \tmpwidth px;
            margin-left: 45px;
        "
    }
    \relscale{1.27}
    \renewcommand\item[4][]{
        \tbfvspace{10pt}
        \iftbftiny \setlength{\parskip}{-10pt} \fi
        \ifx\HCode\undefined
            {\raggedleft{\textsc{\bodyfont{##2}}}\\[1pt]}         % YEAR
        \else
            \HCode{<div class="tex-fill" style="
                display:inline-block;
                width: 100\%;
                text-align: right;
            ">}
            \textsc{\bodyfont{##2}}\\[1pt]
            \HCode{</div>}\\[1pt]
        \fi
        \tbfstyledegree{##1}\\[1pt]                         % Poste
        \tbfstyleinstitution{##3}\\[1pt]                    % Company
        \bodyfontlight{\upshape{##4}}                       % DESCRIPTION
    }
}{
  \end{flat_frame}
}

% Formation
\newenvironmentx{yearlist}[3][1=7.1, 2=0.7, 3=3.9]{
    % Html stuff
    \pgfmathsetmacro{\tmpwidth}{int(#1 * 28.45 * \htmlfactor)}
    \pgfmathsetmacro{\tmpleft}{int(#2 * 28.45 * \htmlfactor)}
    \pgfmathsetmacro{\tmptop}{int(#3 * 28.45 * \htmlfactor)}
    % Append div at top to push me down
    \ifx\HCode\undefined\else
        \HCode{<div class="textcustomblock" style="display: inline-block; width:100\%; overflow: hidden; margin-left: 45px; margin-top: 261px;">\Hnewline</div>\Hnewline}
    \fi
    % Create my space
    \begin{rounded_frame}{\wordeducation}{#1}{#2}{#3}{
        style="
            % No position absolute so I push skill
            display:inline-block;
            float: left;
            max-width: \tmpwidth px;
            margin-left: \tmpleft px;
        "
    }
    \relscale{1.2}
    \renewcommand\item[4][]{
        \textsc{\bodyfont{##2}}                             % YEAR
        \tbfstyledegree{##1}\\[1pt]                         % Diplome
        \tbfstyleinstitution{##3}\\                         % Scholl
        \bodyfontlight{\upshape{##4}}                       % DESCRIPTION
        \tbfvspace{10pt}                                    % Increase horizontal space
    }
}{
    \end{rounded_frame}
}


% Compétence width, x, y
\newenvironmentx{skilllist}[3][1=7.1, 2=0.7, 3=15.2]{
    \pgfmathsetmacro{\tmpwidth}{int(#1 * 28.45 * \htmlfactor)}
    \pgfmathsetmacro{\tmpleft}{int(#2 * 28.45 * \htmlfactor)}
    \pgfmathsetmacro{\tmptop}{int(#3 * 28.45 * \htmlfactor)}
    \begin{rounded_frame}{\wordlanguage}{#1}{#2}{#3}{
        style="
            % No position absolute so I ampushed
            % No margin top so I am under edu
            % Clear both otherwise I push jobs
            display:inline-block;
            clear: both;
            float: left;
            max-width: \tmpwidth px;
            margin-left: \tmpleft px;
        "
    }
    \relscale{1.2}
}{
    \end{rounded_frame}
}


% Language
\newenvironment{languagelist}{
  \renewcommand\item[2]{
    ##1 & ##2 \\
  }
  \begin{tabular}[t]{ll}
}{
  \end{tabular}
}



