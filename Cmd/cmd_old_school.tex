\documentclass[paper=a4, fontsize=11pt]{scrartcl}

\usepackage{xcolor}             % Colors by their 'svgnames'
\usepackage{graphicx}           % include foto
\usepackage{polyglossia}        % french
\setmainlanguage{francais}
\usepackage{amsfonts}           % For checkmark
\usepackage{hyperref}           % for href
\usepackage{url}                % Clickable URL's



\usepackage[bottom=0in,top=1.5in]{geometry}    % Set geometry
    \addtolength{\oddsidemargin}{-0.25in}
    \addtolength{\evensidemargin}{-0.25in}
    \addtolength{\textwidth}{0.5in}
    \addtolength{\textheight}{3in}
    
\usepackage[absolute]{textpos}
    \setlength{\TPHorizModule}{1cm}     % sets our horizontal unit of measuring
    \setlength{\TPVertModule}{1cm}      % sets our vertical unit of measuring
    \textblockorigin{0mm}{0mm}          % and we start measuring in the top left corner

\newlength{\spacebox}
\settowidth{\spacebox}{8888888888}      % Box to align text
\newcommand{\sepspace}{\vspace*{1em}}   % Vertical space macro

\newif\iftbftiny            % My varaible to know if I wanna contract
\let\cvitem\item            % item -> cvitem cause I override item (for retro-compatibility)


\usepackage{sectsty}                % Custom sectioning (see below)
\sectionfont{%                      % Change font of \section command
    \usefont{OT1}{phv}{b}{n}%       % bch-b-n: CharterBT-Bold font
    \sectionrule{0pt}{0pt}{-5pt}{3pt}
}






% New command mission for Rocio: em is the width of a "M"
\providecommand\mission[1]{
    \noindent\makebox[2cm][l]{\textbf{#1}}
}


% A Main part like Education, Formation ... 
\newcommand{\tbfpart}[1]{\section*{\uppercase{#1}}}


% My name
\newcommand{\tbfname}[2]{
    \begin{textblock}{10}(7, 2)
    \centering\huge\bfseries {#1 #2}
    \end{textblock}  
}

% 2 lines describing my mood, speciality
\newcommand{\tbfdescription}[1]{
    #1
}

% Foto
% 1: Path to the image like ../Firgures/Rocio.png
\newcommand{\tbffoto}[1]{ 
    \begin{textblock}{21}(17, 0.2)
    \includegraphics[width=3cm]{#1}
    \end{textblock}  
}


%%%%%%%%%%%%%%%%%%%%%%%%%%%%%%%%%%%%%%%%%%%%%%%%%%%%%%%%%%%%%%%%%%%%%%%%%%%%%%%%%%
%
%            1/ Coordinates
%
%%%%%%%%%%%%%%%%%%%%%%%%%%%%%%%%%%%%%%%%%%%%%%%%%%%%%%%%%%%%%%%%%%%%%%%%%%%%%%%%%%

\newenvironment{coordinatelist}{
    \begin{textblock}{21}(0.5, 0.2)
}
{
    \end{textblock}
}


% Coordinates main function
% 1: Object (tel, email)
% 2: Value (0669696969, toto@pona.fr)
\newcommand{\PersonalEntry}[2]{
    \noindent\hangindent=1em\hangafter=0     % Indentation
    \parbox{\spacebox}{                % Box to align text
    \textit{#1}}                % Entry name (birth, address, etc.)
    #2 \par
}

% Alias
% 1: Value (birthdate, email) 
\newcommand\tbfflag[1]{
    \PersonalEntry{Nat :}{#1}
}
\newcommand\tbfbirth[1]{
    \PersonalEntry{Né le :}{#1}
}
\newcommand\tbfmail[1]{
    \PersonalEntry{Email :}{#1}
}
\newcommand\tbftel[1]{
    \PersonalEntry{Tel :}{#1}
}
\newcommand\tbfaddr[1]{
    \PersonalEntry{Addr :}{#1}
}
\newcommand\tbfvoidpersonal[1]{
    \PersonalEntry{}{#1}
}


%%%%%%%%%%%%%%%%%%%%%%%%%%%%%%%%%%%%%%%%%%%%%%%%%%%%%%%%%%%%%%%%%%%%%%%%%%%%%%%%%%
%
%            2/ Education
%
%%%%%%%%%%%%%%%%%%%%%%%%%%%%%%%%%%%%%%%%%%%%%%%%%%%%%%%%%%%%%%%%%%%%%%%%%%%%%%%%%%


% Education: Yearlist
% Item takes 4 arguments
% 1: Studies (Master de popo)
% 2: Year (1969)
% 3: Place (Ecole de la fete)
% 4: Descritpion (I learnt to smoke and drink)
\newenvironment{yearlist}{
    \tbfpart{Education}
    \renewcommand\item[4][]{
    \vspace{0.5cm}
        \noindent \textbf{##1} \hfill                       % Study
        \colorbox{black}{\color{white}##2} \par             % Duration
        \noindent \textit{##3} \par                         % School
        \noindent\hangindent=2em\hangafter=0 \small ##4     % Description
        \normalsize \par
    }
}


%%%%%%%%%%%%%%%%%%%%%%%%%%%%%%%%%%%%%%%%%%%%%%%%%%%%%%%%%%%%%%%%%%%%%%%%%%%%%%%%%%
%
%            3/ Work
%
%%%%%%%%%%%%%%%%%%%%%%%%%%%%%%%%%%%%%%%%%%%%%%%%%%%%%%%%%%%%%%%%%%%%%%%%%%%%%%%%%%

% Work: joblist
% Item takes 4 arguments, exactely as a yearlist
% 1: Jobname
% 2: Year (1969)
% 3: Place (Ecole de la fete)
% 4: Descritpion (I learnt to smoke and drink)
\newenvironment{joblist}{
    \tbfpart{Work}
    \renewcommand\item[4][]{
    \hangindent=0em\noindent \textbf{##1} \hfill        % Jobname
    \colorbox{black}{\color{white}##2} \par             % Duration
    \hangindent=0em\noindent\textit{##3}\par            % Company
    \hangindent=2em\hangafter=0\small
    {##4 }                                              % Description
    \par
    \hspace{0.5cm}
    }
}


%%%%%%%%%%%%%%%%%%%%%%%%%%%%%%%%%%%%%%%%%%%%%%%%%%%%%%%%%%%%%%%%%%%%%%%%%%%%%%%%%%
%
%            4/ Skills
%
%%%%%%%%%%%%%%%%%%%%%%%%%%%%%%%%%%%%%%%%%%%%%%%%%%%%%%%%%%%%%%%%%%%%%%%%%%%%%%%%%%

% Skill, just like that
\newenvironment{skilllist}{}


% Language: What you spoke
% 1: Language (French)
% 2: Level (native)
\newenvironment{languagelist}{
    \renewcommand\item[2]{
    ##1 & ##2 \\
    }
    \tbfpart{Language}
    % Note: The @{} is to remove indentation
    \begin{tabular}{@{}ll}
}{
    \end{tabular}
}

% 1: Categorie of skill (Logiciels)
% 2: Enumerations of skills (Word, Excel, Python)
\newcommand{\tbfskilltwo}[2]{
    \tbfpart{#1}
    \hangafter=0{#2}
    \par
}
